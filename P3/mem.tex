\documentclass[12pt,a4paper]{article}

\usepackage[utf8]{inputenc} \usepackage[T1]{fontenc} \usepackage{graphicx}
\usepackage{longtable} \usepackage{tabularx} \usepackage{float}
\usepackage{wrapfig} \usepackage{soul} \usepackage{amssymb}
\usepackage{hyperref} \usepackage{caption} \usepackage{subcaption}
\usepackage{pdfpages} \usepackage{sidecap}
\usepackage{amsmath}


\parindent 0in \usepackage[spanish]{babel}
\setlength{\parskip}{0.5\baselineskip} \usepackage{fullpage}
\usepackage{multirow} \usepackage{multicol} \usepackage{framed}
\usepackage{listings} \usepackage{enumerate}

\usepackage{appendix} \usepackage{setspace} \usepackage{amsmath}
\usepackage{mathtools}

%% DEFINICIONES
\newcommand{\TODO}[1]{{\huge \color{red} \textbf{TODO: }#1 }}
\newcommand{\todo}[1]{{\large \color{red} \textbf{TODO: }#1 }}


\title{Pr�ctica 3. \\ Geometr�a computacional} 

\author{Luis Mar�a Costero Valero (lcostero@ucm.es)\\ Jes�s Dom�nech
  Arellano (jdomenec@ucm.es) \\ Jennifer Hern�ndez B�cares (jennhern@ucm.es)}
\date{Marzo 2015}

\begin{document}
\maketitle

\textbf{Demu�strese que el m�ximo de dos polinomios de B�zier $B_{i}^{n}(t)$ con $t\in \left [ 0,1 \right ] $ se alcanza en $t=\frac{i}{n}$.}

Comenzamos escribiendo la definici�n del polinomio de Bernstein i-�simo de grado n: \\
\begin{equation*}
B_{i}^{n}(t) = \binom{n}{i} t^{i}(1-t)^{n-i}
\end{equation*}

Dicha ecuaci�n se corresponde con una distribuci�n binomial, que representa la probabilidad de obtener i caras al lanzar una moneda, siendo la probabilidad de cara igual a $t\in \left [ 0,1\right ] $. \\

Para comprobar que el m�ximo se alcanza en $t=\frac{i}{n}$, derivamos el polinomio de Bernstein e igualamos a 0, con el objetivo de encontrar los extremos y posteriormente ver si son m�ximo y alguno de ellos es $\frac{i}{n}$: \\
\begin{align*}
\frac{\delta B_i^n (t)}{\delta t} &= \binom{n}{i}(it^{i-1}(1-t)^{n-i}-t^{i}(n-i)(1-t)^{n-i-1}) \\
 &= \binom{n}{i}t^{i-1}(1-t)^{n-i-1}(i(1-t)-t(n-i)) \\
 &= 0
\end{align*}

De la ecuaci�n anterior se deduce que se puede tener un m�ximo en los puntos $t=0$, $t=1$ o si $i(1-t)-t(n-i)=0$. Esta �ltima igualdad es equivalente a:
\begin{align*}
i(1-t)=t(n-i) \Longleftrightarrow i-it=tn-it \Longleftrightarrow t=\frac{i}{n}
\end{align*}

\begin{enumerate}
\item Si $t=0$: En este caso, $B_i^n(0)=\binom{n}{i}0^i 1^{n-i}$. Tenemos dos casos dentro de este. Si $i=0$, entonces $B_0^n(0)=1$, que es un m�ximo porque $B_i^n(t)\in \left [ 0,1 \right ]$ para $t\in (0,1)$.Si $i\neq 0$, entonces $B_0^n(0)=0$. Pero si $i=0$, tenemos que $t=\frac{i}{n}=\frac{0}{n}=0$, lo cual es cierto para $t=\frac{i}{n}$ (tercer caso).
\item Si $t=1$: En este caso, $B_i^n(1)=\binom{n}{i}1^i 0^{n-i}$. Volvemos a tener dos casos. Si $n=i$, se tiene que $B_n^n(1) = \binom{n}{n} 1^n 1^0 = 1$. Se trata de un m�ximo, ya que $B_i^n(t) \in \left [ 0,1 \right ]$ para $t\in (0,1)$. Por otro lado, si $n \neq i$, tenemos que $B_i^n(1) = 0$. Sin embargo, si $n=i$, se cumple que $t=\frac{i}{n}=\frac{n}{n} = 1$, que es cierto para $t = 1$ (tercer caso).
\item Por �ltimo, contemplamos el caso $t=\frac{i}{n}$:
\end{enumerate}

\end{document}
