\documentclass[12pt,a4paper]{article}

\usepackage[utf8]{inputenc} \usepackage[T1]{fontenc} \usepackage{graphicx}
\usepackage{longtable} \usepackage{tabularx} \usepackage{float}
\usepackage{wrapfig} \usepackage{soul} \usepackage{amssymb}
\usepackage{hyperref} \usepackage{caption} \usepackage{subcaption}
\usepackage{pdfpages} \usepackage{sidecap}
\usepackage{amsmath}


\parindent 0in \usepackage[spanish]{babel}
\setlength{\parskip}{0.5\baselineskip} \usepackage{fullpage}
\usepackage{multirow} \usepackage{multicol} \usepackage{framed}
\usepackage{listings} \usepackage{enumerate}

\usepackage{appendix} \usepackage{setspace} \usepackage{amsmath}
\usepackage{mathtools}

%% DEFINICIONES
\newcommand{\TODO}[1]{{\huge \color{red} \textbf{TODO: }#1 }}
\newcommand{\todo}[1]{{\large \color{red} \textbf{TODO: }#1 }}


\title{Práctica 3. \\ Geometría computacional} 

\author{Luis María Costero Valero (lcostero@ucm.es)\\ Jesús Doménech
  Arellano (jdomenec@ucm.es) \\ Jennifer Hernández Bécares (jennhern@ucm.es)}
\date{Marzo 2015}

\begin{document}
\maketitle

\textbf{Demuéstrese que el máximo de los polinomios de Bézier $B_{i}^{n}(t)$ con $t\in \left [ 0,1 \right ] $ se alcanza en $t=\frac{i}{n}$.}

Comenzamos escribiendo la definición del polinomio de Bernstein i-ésimo de grado n: \\
\begin{equation*}
  B_{i}^{n}(t) = \binom{n}{i} t^{i}(1-t)^{n-i}
\end{equation*}

Dicha ecuación se corresponde con una distribución binomial, que representa
la probabilidad de obtener i caras al lanzar una moneda, siendo la
probabilidad de cara igual a $t\in \left [ 0,1\right ] $.\\

Para comprobar que el máximo se alcanza en $t=\frac{i}{n}$, derivamos el
polinomio de Bernstein e igualamos a 0, con el objetivo de encontrar los
extremos y posteriormente ver si son máximo y alguno de ellos es
$\frac{i}{n}$:\\

\begin{align*}
  \frac{\delta B_i^n (t)}{\delta t} &= \binom{n}{i}(it^{i-1}(1-t)^{n-i}-t^{i}(n-i)(1-t)^{n-i-1}) \\
                                    &=
                                      \binom{n}{i}t^{i-1}(1-t)^{n-i-1}(i(1-t)-t(n-i))\\
                                    &=\binom{n}{i}t^{i-1}(1-t)^{n-i-1}(i-tn)\\
                                    &= n \left(B^{n-1}_{i-1}(t) - B^{n-1}_{i}(t)\right) \\
                                    &= 0
\end{align*}

De la ecuación anterior se deduce que se puede tener un máximo en los
puntos $t=0$, $t=1$ o si $i(1-t)-t(n-i)=0$. Esta última igualdad es
equivalente a:
\begin{align*}
  i(1-t)=t(n-i) \Longleftrightarrow i-it=tn-it \Longleftrightarrow t=\frac{i}{n}
\end{align*}

\begin{enumerate}
\item Si $t=0$: En este caso, $B_i^n(0)=\binom{n}{i}0^i 1^{n-i}$. Tenemos
  dos casos dentro de este:\\
  Si $i=0$, entonces $B_0^n(0)=1$, que es un
  máximo porque $B_i^n(t)\in \left [ 0,1 \right ]$ para $t\in (0,1)$,
  además se verifica que $t=\frac{i}{n}=\frac{0}{n}=0$, luego verifica el
  tercer caso (el máximo coincide con $t=\frac{i}{n}$).\\
  Si $i\neq 0$, entonces $B_0^n(0)=0$, que no es un máximo.
\item Si $t=1$: En este caso, $B_i^n(1)=\binom{n}{i}1^i 0^{n-i}$. Volvemos
  a tener dos casos:\\
  Si $n=i$, se tiene que $B_n^n(1) = \binom{n}{n} 1^n 0^0 = 1$. Se trata de
  un máximo ya que $B_i^n(t) \in \left [ 0,1 \right ]$ para $t\in
  (0,1)$, además se verifica que $t=\frac{i}{n}=\frac{n}{n} = 1$, luego
  verifica que le máximo coincide con $t=\frac{i}{n}$.\\
  Por otro lado, si $n \neq i$, tenemos que $B_i^n(1) = 0$.
\item Por último, contemplamos el caso $t=\frac{i}{n}$. Sea $i \neq 0, i
  \neq n$, ya que los otros casos corresponden a los apartados
  anteriores. Así, los polinomios de Berstein verifican que $ B_{i}^{n}(t)
  > 0\,\, \forall t \in [0,1]$, y $B_{i}^{n}(0) = 0,\,\,\,
  B_{i}^{n}(1)=0$. Luego es necesario que $t=\frac{i}{n},\,t\neq 0,\,t \neq
  1$ sea un máximo.

\end{enumerate}

\end{document}
