\documentclass[12pt,a4paper]{article}

\usepackage[utf8]{inputenc} \usepackage[T1]{fontenc} \usepackage{graphicx}
\usepackage{longtable} \usepackage{tabularx} \usepackage{float}
\usepackage{wrapfig} \usepackage{soul} \usepackage{amssymb}
\usepackage{hyperref} \usepackage{caption} \usepackage{subcaption}
\usepackage{pdfpages} \usepackage{sidecap}
\usepackage{amsmath} \usepackage{amsthm}


\parindent 0in \usepackage[spanish]{babel}
\setlength{\parskip}{0.5\baselineskip} \usepackage{fullpage}
\usepackage{multirow} \usepackage{multicol} \usepackage{framed}
\usepackage{listings} \usepackage{enumerate}

\usepackage{appendix} \usepackage{setspace} \usepackage{amsmath}
\usepackage{mathtools}

%% DEFINICIONES
\newcommand{\TODO}[1]{{\huge \color{red} \textbf{TODO: }#1 }}
\newcommand{\todo}[1]{{\large \color{red} \textbf{TODO: }#1 }}


\title{Práctica 6. \\ Geometría computacional} 

\author{Luis María Costero Valero (lcostero@ucm.es)\\ Jesús Doménech
  Arellano (jdomenec@ucm.es) \\ Jennifer Hernández Bécares (jennhern@ucm.es)}
\date{}

\begin{document}
\maketitle
\onehalfspace

\begin{center}
  {\large \textbf{Teoría}}
\end{center}
\begin{itemize}
\item \textbf{Dada la secuencia de nodos $t=\mathbb{Z}$, calcúlense las
  funciones B-spline $B_{ik}$ correspondientes}.
\item \textbf{Dada una secuencia de nodos arbitraria t, demuéstrese
  que si p es un polinomio de grado 1 entonces}
  \begin{align*}
    p = \sum_{i}B_{ik}p(t_{i}^{*}),
  \end{align*}
  siendo
  \begin{align*}
    t_{i}^{*}=(t_{i+1}+...+t_{i+k-1})/(k-1).
  \end{align*}
\end{itemize} 

\end{document}
