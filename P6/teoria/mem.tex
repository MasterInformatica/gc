\documentclass[12pt,a4paper]{article}

\usepackage[utf8]{inputenc} \usepackage[T1]{fontenc} \usepackage{graphicx}
\usepackage{longtable} \usepackage{tabularx} \usepackage{float}
\usepackage{wrapfig} \usepackage{soul} \usepackage{amssymb}
\usepackage{hyperref} \usepackage{caption} \usepackage{subcaption}
\usepackage{pdfpages} \usepackage{sidecap}
\usepackage{amsmath} \usepackage{amsthm}


\parindent 0in \usepackage[spanish]{babel}
\setlength{\parskip}{0.5\baselineskip} \usepackage{fullpage}
\usepackage{multirow} \usepackage{multicol} \usepackage{framed}
\usepackage{listings} \usepackage{enumerate}

\usepackage{appendix} \usepackage{setspace} \usepackage{amsmath}
\usepackage{mathtools}

%% DEFINICIONES
\newcommand{\TODO}[1]{{\huge \color{red} \textbf{TODO: }#1 }}
\newcommand{\todo}[1]{{\large \color{red} \textbf{TODO: }#1 }}


\title{Práctica 6. \\ Geometría computacional} 

\author{Luis María Costero Valero (lcostero@ucm.es)\\
  Jesús Doménech Arellano (jdomenec@ucm.es)\\ 
  Jennifer Hernández Bécares (jennhern@ucm.es)} 
\date{}


\newtheorem{observacion}{Observación}

\begin{document}
\maketitle
\onehalfspace

\begin{center}
  {\textbf{Teoría}}
\end{center}

\begin{enumerate}

%%%%%%%%%%%%%%%%%%%%%%%%%%%%%%%%%%%%%%%%%%%%%%%%%%%%%% 


\item Dada la secuencia de nodos $t=\mathbb{Z}$, calcúlense las
  funciones B-spline $B_{ik}$ correspondientes.
  
  Para calcular las funciones B-spline asociadas a la secuencia de
  nodos $t=\mathbb{Z}$, partimos de la recurrencia que define $B_{ik}$
  en el caso general
  \begin{equation}
    B_{ik}=\omega_{ik}B_{i,k-1}+(1-\omega_{i+1,k})B_{i+1,k-1}
  \end{equation}
  donde
  \begin{equation}
    \omega_{ik}(t)=
    \left\lbrace
      \begin{array}{c l}
        \frac{t-t_i}{t_{i+k-1}-t_i}, & $si $t_i\neq t_{i+k-1}\\
        0, & $caso contrario $ \\
      \end{array}
    \right.
  \end{equation}

  Veamos cómo se simplifica la recurrencia al tener en cuenta $t$. En primer lugar
  consideramos que $t_i=i$, ya que $t=\mathbb{Z}$. Además, nunca se va
  a dar que $t_{i}=t_{i+k-1}$, puesto que tenemos como requisito que
  $k>1$ y nos encontramos en $\mathbb{Z}$. La fórmula anterior queda
  de la siguiente forma

  \begin{equation}
    B_{ik}(t)=\frac{t-i}{k-1}B_{i,k-1}(t)+(\frac{k-t+i}{k-1})B_{i+1,k-1}(t)
  \end{equation}

  Por otro lado, queremos demostrar que los B-splines de este tipo son
  traslaciones unos de otros. Esto implica comprobar que
  \begin{equation}
    \label{eq:1}
    B_{ik}(t)=B_{0k}(t-i)
  \end{equation}
  Calcularemos $B_{0k}$ y después
  demostraremos que la igualdad se cumple. \\

  Tomando $i=0$ tenemos

  \begin{equation}
    B_{0k}=\frac{t}{k-1}B_{0,k-1}+\frac{k-t}{k-1}B_{1,k-1}
  \end{equation}
  
  Demostramos (\ref{eq:1}) por inducción:
  \begin{itemize}
  \item \textbf{k=2:}
    \begin{equation*}
      B_{i2}=(t-i)B_{i1}+(2-t+1)B_{i+1,1}=t-i+2-t+i=2
    \end{equation*}
    Por otro lado
    \begin{equation*}
      B_{02}=(t-i)B_{01}+(2-t+1)B_{1,1}=t+2-t=2
    \end{equation*}
    Por tanto, independientemente de dónde evaluemos vemos que
    coinciden $B_{02}$ y $B_{i2}$.
  \item Supongamos ahora por hipótesis que es cierto para $k=k-1$, es
    decir, que $B_{i,k-1}(t)=B_{0,k-1}(t-i)$. En este caso, tendríamos
    también que $B_{i+1,k-1}(t)=B_{1,k-1}(t-i)$. Usaremos esto para
    continuar la demostración.
    
  \item Comprobamos que se cumple para \textbf{$B_{ik}$}:
    \begin{equation}
      \begin{array}{l c l}
        B_{ik}(t)&=&\frac{t-i}{k-1}B_{i,k-1}(t)+\frac{k-t-i}{k-1}B_{i+1,k-1}(t) \\
                 &\underset{(HI)}{=}&\frac{t-i}{k-1}B_{0,k-1}(t-i)+\frac{k-t-i}{k-1}B_{1,k-1}(t-i) \\
                 &=&B_{0k}(t-i)
      \end{array}
    \end{equation}
    
  \end{itemize}
  \newpage{}
  %%%%%%%%%%%%%%%%%%%%%%%%%%%%%%%%%%%%%%%%%%%%%%%%%%%%%% 
\item Dada una secuencia de nodos arbitraria t, demuéstrese
  que si p es un polinomio de grado 1 entonces
  \begin{align*}
    p = \sum_{i}B_{ik}p(t_{i}^{*}),
  \end{align*}
  siendo
  \begin{align*}
    t_{i}^{*}=(t_{i+1}+...+t_{i+k-1})/(k-1).
  \end{align*}

  Consideramos la identidad de Marsden, dada por:
  $$(x-t)^{k-1}=\sum\limits_{i=1}^{n}B_{i,k}(t)\cdot\psi_{i,k}(x),\,\,\,\,\,\,\,\,\,\,\,\,\,\,\,\, \forall x\in\mathbb{R},\,\,\,\, t_k\le t \le t_{n+1}$$
  $\,\,\,\,\psi_{i,k}(x)=(x-t_{i+1})\dots (x-t_{i+k-1})$

  Si dividimos entre $(k-1)!$ obtenemos:
  $$\frac{(x-t)^{k-1}}{(k-1)!}=\sum\limits_{i=1}^{n}B_{i,k}(t)\cdot\frac{\psi_{i,k}(x)}{(k-1)!}$$
  Derivando una vez respeccto a $x$:
  $$\frac{(x-t)^{k-2}}{(k-2)!}=\sum\limits_{i=1}^{n}B_{i,k}(t)\cdot\frac{\psi_{i,k}'(x)}{(k-1)!}$$,
  y en general derivando $m-1$ veces:
  $$\frac{(x-t)^{k-m}}{(k-m)!}=\sum\limits_{i=1}^{n}B_{i,k}(t)\cdot\frac{\psi_{i,k}^{(m-1)}(x)}{(k-1)!}$$


  consideremos ahora el caso en el que $m=k-1$, sustituyendo $m$ y
  aplicando la observación \ref{observacion:1} se tiene
  que:
  $$(x-t)=\sum\limits_{i=1}^{n}B_{i,k}(t)\cdot\left[x-\frac{t_{i+1}+\dots+t_{i+k-1}}{(k-1)}\right]$$

  Lo que implica que para cualquier polinomio de la forma
  $p(t)=a \cdot t + b$, y siendo
  $t_{i}^{*}=\frac{t_{i+1}+\dots+t_{i+k-1}}{k-1}$ se tenga lo buscado:
  $$p(t)=\sum\limits_{i=1}^{n}B_{i,k}(t)p(t_{i}^{*})$$

  \vspace{1cm}

  \begin{observacion}
    \label{observacion:1}
    Sea $f=(x-t_1)\dots(x-t_{n})$, entonces se verifica
    que $$f^{(n-1)}=x-\left((n-1)!\cdot(t_1+\dots+t_{n})\right)$$
  \end{observacion}
    Operando $f$ se tiene un polinomio de grado $n$ que se puede expresar
    como\\$f=x^{n}-x^{n-1}(t_1+\cdots+t_{n})+q$\,\,\,\,donde $q$ es un polinomio de
    grado $n-1$.

    Si se deriva $n-1$ veces, se obtiene el resutlado buscado:
    $$f^{(n-1)}=x-\left((n-1)!\cdot(t_1+\dots+t_{n})\right)$$

\end{enumerate} 
\end{document}
