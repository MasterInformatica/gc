\documentclass[12pt,a4paper]{article}

\usepackage[utf8]{inputenc} \usepackage[T1]{fontenc} \usepackage{graphicx}
\usepackage{longtable} \usepackage{tabularx} \usepackage{float}
\usepackage{wrapfig} \usepackage{soul} \usepackage{amssymb}
\usepackage{hyperref} \usepackage{caption} \usepackage{subcaption}
\usepackage{pdfpages} \usepackage{sidecap}
\usepackage{amsmath} \usepackage{amsthm}


\parindent 0in \usepackage[spanish]{babel}
\setlength{\parskip}{0.5\baselineskip} \usepackage{fullpage}
\usepackage{multirow} \usepackage{multicol} \usepackage{framed}
\usepackage{listings} \usepackage{enumerate}

\usepackage{appendix} \usepackage{setspace} \usepackage{amsmath}
\usepackage{mathtools}

%% DEFINICIONES
\newcommand{\TODO}[1]{{\huge \color{red} \textbf{TODO: }#1 }}
\newcommand{\todo}[1]{{\large \color{red} \textbf{TODO: }#1 }}


\title{Práctica 6. \\ Geometría computacional} 

\author{Luis María Costero Valero (lcostero@ucm.es)\\ Jesús Doménech
  Arellano (jdomenec@ucm.es) \\ Jennifer Hernández Bécares (jennhern@ucm.es)}
\date{}

\begin{document}
\maketitle
\onehalfspace

\begin{center}
  {\large \textbf{Teoría}}
\end{center}
\begin{itemize}
\item Dada la secuencia de nodos $t=\mathbb{Z}$, calcúlense las
  funciones B-spline $B_{ik}$ correspondientes.

  Tomamos en primer lugar una secuencia no decreciente de nodos
  $t=(t_i)$. Los B-splines de orden 1 asociados a la secuencia
  anterior de nodos vienen dados por las funciones características de
  esta partición, que son continuas por la derecha:
  
  \begin{equation}
    B_{ik}(t)=
    \left\lbrace
    \begin{array}{c l}
      1, & $si $t_i\le t<t_{i+1}\\
      0, & $caso contrario $ \\
    \end{array}
    \right.
  \end{equation}

  La ecuación anterior cumple que $\sum_{i}B_{i1}(t)=1$ para todo $t$. Además, si los nodos $t_i$ y $t_{i+1}$ coinciden, se tiene que $B_{i1}=0$. \\

  Para obtener los B-splines de órdenes más altos, nos basamos en la siguiente recurrencia:
  \begin{equation}
    B_{ik}=\omega_{ik}B_{i,k-1}+(1-\omega_{i+1,k})B_{i+1,k-1}
  \end{equation}
  donde
      
  \begin{equation}
    \omega_{i1}(t)=
    \left\lbrace
    \begin{array}{c l}
      \frac{t-t_i}{t_{i+k-1}-t_i}, & $si $t_i\neq t_{i+k-1}\\
      0, & $caso contrario $ \\
    \end{array}
    \right.
  \end{equation}

  Desarrollando las ecuaciones anteriores, tendríamos que el B-spline
  de segundo orden viene dado por
  \begin{equation}
    B_{i2}=\omega_{i2}B_{i1}+(1-\omega_{i+1,2})B_{i+1,1}
  \end{equation}

  Seguimos desarrollando para sacar la fórmula general de
  $B_{ik}$. Para ello, calculamos el B-spline de tercer orden
  \begin{equation}
    \begin{array}{c l}
    B_{i3} &= \omega_{i3}B_{i2}+(1-\omega_{i+1,3})B_{i+1,2} \\
    &= \omega_{i3}(\omega_{i2}B_{i1}+(1-\omega_{i+1,2})B_{i+1,1})+(1-\omega_{i+1,3})B_{i+1,2}
    \end{array}
  \end{equation}

  Por otro lado tenemos que
  \begin{equation*}
     B_{i+1,2}=\omega_{i+1,2}B_{i+1,1}+(1-\omega_{i+2,2})B_{i+2,1}
  \end{equation*}
  por lo que, sustituyendo, tenemos lo siguiente
  \begin{equation}
    \begin{array}{r l}
      B_{i3} = &\omega_{i3}(\omega_{i2}B_{i1} + (1-\omega_{i+1,2})B_{i+1,1}) + \\
      & +(1-\omega_{i+1,3})(\omega_{i+1,2}B_{i+1,1}+(1-\omega_{i+2,2})B_{i+2,1}) \\
      =&\omega_{i3}\omega_{i2}B_{i1}+(\omega_{i3}(1-\omega_{i+1,2})+(1-\omega_{i+1,3})\omega_{i+1,2})B_{i+1,1} \\
      & + (1-\omega_{i+1,3})(1-\omega_{i+2,2})B_{i+2,1}
    \end{array}
  \end{equation}

  En la ecuación anterior vemos que $B_{i3}$ es función de $B_{i1}$,
  $B_{i+1,1}$ y $B_{i+2,1}$. Esto se puede generalizar, pues $B_{ik}$
  siempre va a depender de $B_{i,1},...,B_{i+k-1,1}$. Por otro lado,
  estas $B_{ik}$ están multiplicadas por ciertos polinomios en $t$ que,
  basándonos en la definición de $\omega_{i,k}$ (en la cual aparece
  $t$ en el numerador de la fracción) y sabiendo que vamos
  a hacer como máximo $k-1$ multiplicaciones, tendrán grado como
  mucho $k-1$. Por tanto, tenemos que los $B$-splines de orden $k$
  cumplen lo siguiente
  \begin{equation}
    B_{ik}=\sum_{j=i}^{i+k-1}b_{jk}B_{j1}
  \end{equation}
  donde hemos llamado $b_{jk}$ al polinomio de grado menor o igual que
  $k-1$ que multiplica a cada una de las $B_{j1}$. Además, viendo la forma de $B_{i3}$, podemos decir que el
  polinomio tendrá un grado exactamente igual a $k-1$, ya que al
  realizar la multiplicación no se va a anular ninguno de los términos
  y se mantendrá el grado obtenido.

  %%%%%%%%%%%%%%%%%%%%%%%%%%%%%%%%%%%%%%%%%%%%%%%%%%%%%%
\item Dada una secuencia de nodos arbitraria t, demuéstrese
  que si p es un polinomio de grado 1 entonces
  \begin{align*}
    p = \sum_{i}B_{ik}p(t_{i}^{*}),
  \end{align*}
  siendo
  \begin{align*}
    t_{i}^{*}=(t_{i+1}+...+t_{i+k-1})/(k-1).
  \end{align*}
\end{itemize} 

\end{document}
