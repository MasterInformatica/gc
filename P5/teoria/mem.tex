\documentclass[12pt,a4paper]{article}

\usepackage[utf8]{inputenc} \usepackage[T1]{fontenc} \usepackage{graphicx}
\usepackage{longtable} \usepackage{tabularx} \usepackage{float}
\usepackage{wrapfig} \usepackage{soul} \usepackage{amssymb}
\usepackage{hyperref} \usepackage{caption} \usepackage{subcaption}
\usepackage{pdfpages} \usepackage{sidecap}
\usepackage{amsmath} \usepackage{amsthm}


\parindent 0in \usepackage[spanish]{babel}
\setlength{\parskip}{0.5\baselineskip} \usepackage{fullpage}
\usepackage{multirow} \usepackage{multicol} \usepackage{framed}
\usepackage{listings} \usepackage{enumerate}

\usepackage{appendix} \usepackage{setspace} \usepackage{amsmath}
\usepackage{mathtools}

%% DEFINICIONES
\newcommand{\TODO}[1]{{\huge \color{red} \textbf{TODO: }#1 }}
\newcommand{\todo}[1]{{\large \color{red} \textbf{TODO: }#1 }}


\title{Práctica 5. \\ Geometría computacional} 

\author{Luis María Costero Valero (lcostero@ucm.es)\\ Jesús Doménech
  Arellano (jdomenec@ucm.es) \\ Jennifer Hernández Bécares (jennhern@ucm.es)}
\date{}

\begin{document}
\maketitle
\doublespace

\textbf{Si $f=gh$, demuéstrese la fórmula de Leibniz para las diferencias divididas,}
\begin{equation*}
  [\tau_i,\dots,\tau_{i+k}]f=\sum\limits_{r=i}^{i+k}([\tau_i,\dots,\tau_r]g)([\tau_r,\dots,\tau_{i+k}]h).
\end{equation*}

Sea P el polinomio interpolador de f en los puntos $\tau_i, ... ,
\tau_{i+k}$. Entonces se verifica que $$[\tau_{i}, ..., \tau_{i+k}](Pg) =
[\tau_i, ..., \tau_{i+k}](fg)$$
Además, el polinomio interpolador de Newton viene dado por:
$$ P = \sum\limits^{i+k}_{r=i}(x-\tau_i)\dots(x-\tau_{r-1})[\tau_i,\dots,\tau_{i+k}]f$$
Luego se obtiene que:
\begin{align*}
  [\tau_i,\dots,\tau_{i+k}](fg) &= [\tau_i,\dots,\tau_{i+k}](Pg) = (Def.\,polinom.\,Newton)=\\
                      &= [\tau_i,\dots,\tau_{i+k}]\left(\sum\limits_{r=i}^{i+k}(x-\tau_i)...(x-\tau_{r-1})[\tau_i,...,\tau_r]f
                        \cdot g\right)
\end{align*}
Como se cumple la siguiente propiedad:
$$[x_0, ..., x_i, y_0, .., y_j]((t-x_0)...(t-x_i)f) = [y_0,...,y_j]f$$\\
Y operando se obtiene el resultado buscado:
\begin{align*}
  [\tau_i,\dots,\tau_{i+k}](fg) &= [\tau_i,\dots,\tau_{i+k}]\left(\sum\limits_{r=i}^{i+k}(x-\tau_i)...(x-\tau_{r-1})[\tau_i,...,\tau_r]f
                        \cdot g\right)\\
                      &=\sum\limits^{i+k}_{r=i}[\tau_i,\dots,\tau_{r}]f \cdot
                        [\tau_i,\dots,\tau_{i+k}]\left((x-\tau_i,\dots,\tau_{r-1})g\right)=\\
                      &=\sum\limits^{i+k}_{r=i}[\tau_i,\dots,\tau_r]f \cdot [\tau_r,
                        \dots, \tau_{i+k}]g\\
\end{align*}\\

\textbf{Aplíquese esta fórmula e inducción al caso particular $g(x)=x,$, $h(x)=1/x$ para deducir que}
\begin{equation*}
  [\tau_1,...,\tau_n](1/x)=(-1)^{n-1}/(\tau_1...\tau_n).
\end{equation*}

Consideramos $f(x)=g(x)h(x)$, es decir $f(x)=1$. Vamos a comprobar las
siguientes propiedades:
\begin{enumerate}
\item 
  \begin{itemize}
  \item $[\tau_i]f=f(\tau_i)=1$
  \item $[\tau_i, \tau_{i+1}, \dots, \tau_j]f = 0$. Por inducción sobre el
    número de elementos.\\
    \textit{Caso base:} $j=i+1$, se tiene que $[\tau_i,\tau_{i+1}]f =
    \frac{[\tau_{i+1}]f -[\tau_i]f}{\tau_{i+2}-\tau_{i}} = \frac{1 -
      1}{\tau_{i+1} - \tau_i} = 0.$\\
    \textit{Hipótesis de inducción:} Supongamos que se cumple para $j=n-1>i+1$, y
    demostremos que se verifica para $j=n$.\\
    $[\tau_i, \tau_{i+1},\dots,\tau_{n}] f =
    \frac{[\tau_{i+1},\dots,\tau_n]f -
      [\tau_i,\dots,\tau_{n-1}]f}{\tau_n-\tau_i} \stackrel{H.I.}{=} \frac{0-0}{\tau_n -
      \tau_i}=0$
  \end{itemize}
\item 
  \begin{itemize}
  \item $[\tau_i]g=g(\tau_i)=\tau_i$
  \item $[\tau_i,\tau_{i+1}]g=\frac{[\tau_{i+1}]g-[\tau{i}]g}{\tau_{i+1}-\tau_{i}}=1$
  \item $[\tau_i, \tau_{i+1}, \tau_{i+2}, \dots, \tau_j]g = 0$. Por inducción sobre el
    número de elementos.\\
    \textit{Caso base:} $j=i+2$, se tiene que $[\tau_i,\tau_{i+1},\tau_{i+2}]g =
    \frac{[\tau_{i+1},\tau_{i+2}]g -[\tau_i,\tau_{i+1}]g}{\tau_{i+2}-\tau_{i}} = \frac{1 -
      1}{\tau_{i+1} - \tau_i} = 0.$\\
    \textit{Hipótesis de inducción:} Supongamos que se cumple para $j=n-1>i+2$, y
    demostremos que se verifica para $j=n$.\\
    $[\tau_i, \tau_{i+1}, \tau_{i+2},\dots,\tau_{n}] g =
    \frac{[\tau_{i+1}, \tau_{i+2},\dots,\tau_n]g -
      [\tau_i,\tau_{i+1},\tau_{i+2},\dots,\tau_{n-1}]g}{\tau_n-\tau_i} \stackrel{H.I.}{=} \frac{0-0}{\tau_n -
      \tau_i}=0$
  \end{itemize}
\end{enumerate}

Vamos a comprobar ahora el resultado del enunciado por inducción sobre el
número de elementos $\tau$.\\
\textit{Caso base} $n=1$: Entonces:
$$ 1 = [\tau_1]f \stackrel{Leibniz}{=} ([\tau_1]g)([\tau_1]h) = \tau_1 \cdot
([\tau_1]h) \Longrightarrow [\tau_1]h=[\tau_1](1/x) = 1/\tau_1 =
(-1)^{1-1}/\tau_1$$
Supongamos cierto para $n=j-1$, y comprobemos para $n=j$
\begin{align*}
  0&\stackrel{(1)}{=}[\tau_{1},\dots,\tau_{j}]f\stackrel{Leibniz}{=}\sum\limits^{j}_{r=1}([\tau_1,\dots,\tau_r]g)([\tau_r,\dots,\tau_j]h)=\\
   &=([\tau_1]g)([\tau_1,\dots,\tau_j]h) +
     ([\tau_1,\tau_2]g)([\tau_2,\dots,\tau_j]h) +
     \sum\limits^{j}_{r=3}([\tau_1,\dots,\tau_r]g)(\tau_r,\dots,\tau_j]h)=\\
   &\stackrel{(2)}{=} \tau_1\cdot[\tau_1,\dots,\tau_j]h + [\tau_2,\dots,\tau_j]h 
\end{align*}
Por otro lado se tiene que:\\
$[\tau_1,\dots,\tau_j]h = \frac{[\tau_2,\dots,\tau_j]h -
  [\tau_1,\dots,\tau_{j-1}]h}{\tau_j-\tau_1} \Longrightarrow
[\tau_2,\dots,\tau_j]h = (\tau_j-\tau_1)[\tau_{1},\dots,\tau_{j}]h +
[\tau_1,\dots,\tau_{j-1}]h$\\
Y aplicando la hipótesis de inducción sobre $[\tau_1,\dots,\tau_{j-1}]h$ se
obtiene:\\
$[\tau_2,\dots,\tau_j]h = (\tau_j-\tau_1)[\tau_1,\dots,\tau_j]h +
(-1)^{j-2}/(\tau_1\dots\tau_{j-1})$\\
E introduciendo esta expresión en la expresión anterior, obtenemos lo
buscado:
\begin{align*}
  0 = [\tau_1,\dots,\tau_j]f&=\tau_1\cdot[\tau_1,\dots,\tau_j]h +
                              [\tau_2,\dots,\tau_j]h=\\
                            &=\tau_1\cdot[\tau_1,\dots,\tau_j]h + (\tau_j-\tau_1)[\tau_1,\dots,\tau_j]h +
(-1)^{j-2}/(\tau_1\dots\tau_{j-1}) =\\
  &=(\tau_1+\tau_j-\tau_1)[\tau_1,\dots,\tau_j]h +
    (-1)^{j-2}/(\tau_1\dots\tau_{j-1}) \Longrightarrow\\
[\tau_1,\dots,\tau_j](1/x) &= (-1)^{j-2}/(\tau_1\dots\tau_{j-1}) \cdot
                             (-1)/\tau_j=\\
                             &=(-1)^{j-1}/(\tau_1\dots\tau_{j})
\end{align*}
\end{document}
