\documentclass[12pt,a4paper]{article}

\usepackage[utf8]{inputenc} \usepackage[T1]{fontenc} \usepackage{graphicx}
\usepackage{longtable} \usepackage{tabularx} \usepackage{float}
\usepackage{wrapfig} \usepackage{soul} \usepackage{amssymb}
\usepackage{hyperref} \usepackage{caption} \usepackage{subcaption}
\usepackage{pdfpages} \usepackage{sidecap}
\usepackage{amsmath} \usepackage{amsthm}


\parindent 0in \usepackage[spanish]{babel}
\setlength{\parskip}{0.5\baselineskip} \usepackage{fullpage}
\usepackage{multirow} \usepackage{multicol} \usepackage{framed}
\usepackage{listings} \usepackage{enumerate}

\usepackage{appendix} \usepackage{setspace} \usepackage{amsmath}
\usepackage{mathtools}

%% DEFINICIONES
\newcommand{\TODO}[1]{{\huge \color{red} \textbf{TODO: }#1 }}
\newcommand{\todo}[1]{{\large \color{red} \textbf{TODO: }#1 }}


\title{Práctica 5. \\ Geometría computacional} 

\author{Luis María Costero Valero (lcostero@ucm.es)\\ Jesús Doménech
  Arellano (jdomenec@ucm.es) \\ Jennifer Hernández Bécares (jennhern@ucm.es)}
\date{}

\begin{document}
\maketitle

\textbf{Si $f=gh$, demuéstrese la fórmula de Leibniz para las diferencias divididas,}
\begin{equation*}
  [\tau_i,\dots,\tau_{i+k}]f=\sum\limits_{r=i}^{i+k}([\tau_i,\dots,\tau_r]g)([\tau_r,\dots,\tau_{i+k}]h).
\end{equation*}

Sea P el polinomio interpolador de f en los puntos $\tau_i, ... ,
\tau_{i+k}$. Entonces se verifica que $$[\tau_{i}, ..., \tau_{i+k}](Pg) =
[\tau_i, ..., \tau_{i+k}](fg)$$
Además, el polinomio interpolador de Newton viene dado por:
$$ P = \sum\limits^{i+k}_{r=i}(x-\tau_i)\dots(x-\tau_{r-1})[\tau_i,\dots,\tau_{i+k}]f$$
Luego se obtiene que:
\begin{align*}
  [\tau_i,\dots,\tau_{i+k}](fg) &= [\tau_i,\dots,\tau_{i+k}](Pg) = (Def.\,polinom.\,Newton)=\\
                      &= [\tau_i,\dots,\tau_{i+k}]\left(\sum\limits_{r=i}^{i+k}(x-\tau_i)...(x-\tau_{r-1})[\tau_i,...,\tau_r]f
                        \cdot g\right)
\end{align*}
Como se cumple la siguiente propiedad:
$$[x_0, ..., x_i, y_0, .., y_j]((t-x_0)...(t-x_i)f) = [y_0,...,y_j]f$$\\
Y operando se obtiene el resultado buscado:
\begin{align*}
  [\tau_i,\dots,\tau_{i+k}](fg) &= [\tau_i,\dots,\tau_{i+k}]\left(\sum\limits_{r=i}^{i+k}(x-\tau_i)...(x-\tau_{r-1})[\tau_i,...,\tau_r]f
                        \cdot g\right)\\
                      &=\sum\limits^{i+k}_{r=i}[\tau_i,\dots,\tau_{r}]f \cdot
                        [\tau_i,\dots,\tau_{i+k}]\left((x-\tau_i,\dots,\tau_{r-1})g\right)=\\
                      &=\sum\limits^{i+k}_{r=i}[\tau_i,\dots,\tau_r]f \cdot [\tau_r,
                        \dots, \tau_{i+k}]g\\
\end{align*}
\newpage{}
\textbf{Aplíquese esta fórmula e inducción al caso particular $g(x)=x,$, $h(x)=1/x$ para deducir que}
\begin{equation*}
  [\tau_1,...,\tau_n](1/x)=(-1)^{n-1}/(\tau_1...\tau_n).
\end{equation*}

\end{document}
