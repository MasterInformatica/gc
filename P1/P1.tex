\documentclass{article}
\usepackage[utf8]{inputenc}
\usepackage[spanish]{babel}
\usepackage{enumitem}

\title{Práctica 1. \\ Geometría computacional}
\author{Luis María Costero Valero \\ Jesús Doménech Arellano \\ Jennifer Hernández Bécares}
\date{Febrero 2015}

\begin{document}

\maketitle

\begin{enumerate}
\item Estúdiese si las siguientes curvas son equivalentes utilizando el algoritmo o de manera directa. 
\begin{enumerate}
\item $\gamma(t)=(t-1,t),I=(0,1)$ \\ $\bar{\gamma}(t)=(2t-5,3-t),\bar{I}=(-1,0)$ \\
Ambas signaturas son (0,0), por tanto podemos asegurar (a mano y mediante el algoritmo) que las curvas son equivalentes.
\item $\gamma(t)=(2\cos t,3\sin t),I=(0,2\pi)$ \\ $\bar{\gamma}(t)=(3\cos t,2\sin t),\bar{I}=(0,2\pi)$ \\
El algoritmo nos devuelve que las curvas son equivalentes. 
\item $\gamma(t)=(t,\frac{1}{2t}),I=(1/10, 10)$ \\ $\bar{\gamma}(t)=(\cosh t,\sinh t),\bar{I}=(0,1)$ \\
\item $\gamma(t)=(t,t^{2}),I=(-2,2)$ \\
REVISAR. Las curvas no son equivalentes en el intervalo dado. 
$\bar{\gamma}(t)=(-\frac{1}{2}\sqrt{3}\log(t)^{2}+\frac{1}{2}\log(t)+1,\frac{1}{2}\log(t)^{2}+\frac{1}{2}\sqrt{3}\log(t)-1),\bar{I}=(1/10,10)$ \\
REVISAR. Sale not a number
\end{enumerate}

\item Estúdiese si la signatura de una curva puede tener alguna de las siguientes gráficas, dando ejemplos o argumentando si no es posible:
\begin{enumerate}[label=\roman*]
\item Si tomamos la curva $\gamma(t)=(\cos t, \sin t)$ $I=(0,2\pi)$, la signatura toma el valor $(1,0)$, que tiene la forma que se pedía. Para cualquier otra circunferencia, también se toma la forma $($cte$,0)$.   
\item segunda
\item tercera
\item cuarta
\end{enumerate}

\item ¿Qué sucede con la signatura de una curva si la curva es simétrica respecto de una recta?

\item Si se invierte la orientación de recorrido de una curva, ¿qué ocurre con su signatura?
\end{enumerate}

\end{document}
