\documentclass[12pt,a4paper]{article}

\usepackage[utf8]{inputenc} \usepackage[T1]{fontenc} \usepackage{graphicx}
\usepackage{longtable} \usepackage{tabularx} \usepackage{float}
\usepackage{wrapfig} \usepackage{soul} \usepackage{amssymb}
\usepackage{hyperref} \usepackage{caption} \usepackage{subcaption}
\usepackage{pdfpages} \usepackage{sidecap}


\parindent 0in \usepackage[spanish]{babel}
\setlength{\parskip}{0.5\baselineskip} \usepackage{fullpage}
\usepackage{multirow} \usepackage{multicol} \usepackage{framed}
\usepackage{listings} \usepackage{enumerate}

\usepackage{appendix} \usepackage{setspace}


%% DEFINICIONES
\newcommand{\TODO}[1]{{\huge \color{red} \textbf{TODO: }#1 }}
\newcommand{\todo}[1]{{\large \color{red} \textbf{TODO: }#1 }}


\title{Práctica 1. \\ Geometría computacional} 

\author{Luis María Costero Valero (lcostero@ucm.es)\\ Jesús Doménech
  Arellano (jdomenec@ucm.es) \\ Jennifer Hernández Bécares (jennhern@ucm.es)}
\date{Febrero 2015}

\begin{document}
\maketitle
\onehalfspace

\begin{enumerate}
\item Estúdiese si las siguientes curvas son equivalentes utilizando el
  algoritmo o de manera directa.
  \begin{enumerate}
  \item $\gamma(t)=(t-1,t),I=(0,1)$ \\
    $\bar{\gamma}(t)=(2t-5,3-t),\bar{I}=(-1,0)$ \\ Ambas signaturas son
    (0,0), por tanto podemos asegurar (a mano y mediante el algoritmo) que
    las curvas son equivalentes. \\
  \item $\gamma(t)=(2\cos t,3\sin t),I=(0,2\pi)$ \\
    $\bar{\gamma}(t)=(3\cos t,2\sin t),\bar{I}=(0,2\pi)$ \\ 
    Se trata de dos elipses una sobre el eje $x$ y la otra sobre el
    eje $y$.\\
    Al ejecutar el programa obtenemos que las curvas son
    equivalentes, tal y como esperábamos. 
  \item $\gamma(t)=(t,\frac{1}{2t}),I=(1/10, 10)$ \\
    $\bar{\gamma}(t)=(\cosh t,\sinh t),\bar{I}=(0,1)$ \\
    Aplicando el algoritmo obtenemos las siguientes signaturas:\\
    $sig_{\gamma}(t)) = (\frac{8}{t^{3}(4 + t^{-4})^{\frac{3}{2}}},
    \frac{t^{4}(-192t^{4} + 48)}{(4t^{4} + 1)^{3}})$ \\
    $sig_{\bar{\gamma}}(t) = (\frac{-1}{(\cosh
      2t)^{\frac{3}{2}}},\frac{12\sinh 2t}{3\cosh 2t + \cosh 6t})$ \\
    Cuyas imágenes no coinciden en los intervalos $I$ e $\bar{I}$. Tal
    y como indica el programa.
  \item $\gamma(t)=(t,t^{2}),I=(-2,2)$ 
    $\bar{\gamma}(t)=(-\frac{1}{2}\sqrt{3}\log(t)^{2}+\frac{1}{2}\log(t)+1,\frac{1}{2}\log(t)^{2}+\frac{1}{2}\sqrt{3}\log(t)-1),\bar{I}=(1/10,10)$
    \\Igual que en el apartado anterior, aplicando el algoritmo
    obtenemos las siguientes signaturas:\\
    \\$sig_{\gamma}(t) = (\frac{2}{(4t^{2} + 1)^{\frac{3}{2}}},
    \frac{-24t}{(4t^{2} + 1)^3})$
\\$sig_{\bar{\gamma}}(t) = (\frac{2}{t(\frac{4\log^2 t +
    1}{t^2})^{\frac{1}{2}}(4\log^2 t + 1)}, \frac{-96\log t}{256\log^6
  t + 192\log^4 t + 48\log^2 t + 4})$
\\Cuyas imágenes no coinciden en los intervalos $I$ e $\bar{I}$. Tal
    y como indica el programa.
  \end{enumerate}

\item Estúdiese si la signatura de una curva puede tener alguna de las
  siguientes gráficas, dando ejemplos o argumentando si no es posible:
  \begin{enumerate}
  \item[i.]\textbf{$sig(t)=(k_{0},0), k_{0}\in \mathbb{R}$ }. Consideramos
    la curva $\gamma(t)=(\cos t, \sin t)$ $I=(0,2\pi)$, la signatura toma
    el valor $(1,0)$, que tiene la forma que se pedía en el enunciado. Para
    cualquier otra circunferencia, también se verifica que la signatura
    toma la forma $($cte$,0)$.
  \item[ii.]\textbf{$sig(t)=(k_{0}, k_{s}),\,\,
      k_{0},k_{s}\in\mathbb{R},\,\, k_{s}\ne0$}.Vamos
    a comprobar que no es posible este caso. Sea $\gamma(t)$ una curva con
    curvatura $K=k_{0}$. Se tiene que
    $\frac{dK}{ds} = \frac{dK}{dt} \frac{1}{||\gamma'||} =
    \frac{dk_{0}}{dt} \frac{1}{||\gamma'||} = 0 * \frac{1}{||\gamma'||} =
    0$. Luego $sig_{\gamma}(t) = (k_{0}, 0) \ne (k_{0}, k_{s})$.
  \item[iii.]\textbf{$sig(t)=(f(t),f(t))$}. Sea $\gamma$ una curva
    p.p.a. con signatura $sig_{\gamma}(t)=(f(t),f(t))$. Por la definición
    de signatura se tiene $K_{\gamma}=f$ y $\frac{dK_{\gamma}}{ds}=f$. Por
    estar parametrizada por la longitud del arco
    $\frac{1}{||\gamma'||} = 1$, luego
    $f = \frac{dK_{\gamma}}{ds} = \frac{dK_{\gamma}}{dt}
    \frac{1}{||\gamma'||} = \frac{dK_{\gamma}}{dt} = \frac{df}{dt}$.
    Resolviendo la ecuación $f=e^{t}=K_{\gamma}$, y por el teorema
    fundamental de curvas se sabe que va a existir una curva regular con
    esta curvatura, y por tanto con esta signatura.
  \item[iv.]\textbf{$sig(t)=(k_{0}, f(t)),\,\, k_{0}\in\mathbb{R}$}. Sea
    una curva $\gamma$ tal que su survatura es $K_{\gamma} = k_{0}$.
    $\frac{dK_{\gamma}}{ds} = \frac{dK_{\gamma}}{dt} \frac{1}{||\gamma'||}
    = \frac{dk_{0}}{dt} \frac{1}{||\gamma'||} = 0$.
    Por lo que es imposible que exista una curva con signatura igual a la
    del enunciado.
  \end{enumerate}

\item ¿Qué sucede con la signatura de una curva si la curva es simétrica
  respecto de una recta? \\
  ---------------------------------------------------------------------------\\
  \todo{La idea de Jesús puede tener sentido. No hay que olvidar de poner
    tanto la $\gamma$ como la curvatura y la diferencial en función de
    (a+b-t) en vez de (t) para que se vayan anunlando signos por el
    camino. Con eso creo que debería bastar para que salga simétrica
    respecto al eje OY}
  ---------------------------------------------------------------------------\\
  Una curva cualquiera mantiene la signatura a por movimientos rigidos
  directos (rotaiones y traslaciones), ouego se puede mover para que sea
  simétrica respecto del eje X, es decir gamma(barra)=(g1,g2)=(g1, -g2).
  Se calcula la signautra de gamma(barra) y sale lo que sea. Jesús dice que
  sale de signo contrario.

  %% 4
\item Si se invierte la orientación de recorrido de una curva, ¿qué ocurre
  con su signatura?\\
  ---------------------------------------------------------------------------\\
%  \todo{OJO, si se pone que $\bar{\gamma}(t) = \gamma(a+b-t)$ entonces creo
%    que sale la segunda coordenada de la signatura igual. Probar a poner
%    tanto la K, como la diferenciales en función de a+b-t, al derivar ese t
%    anulará al signo contrario de la curvatura, y quedará la segunda
%    cordenada igual.}
  \todo{REVISAR: El formato es un poco feo, habría que colocar las cosas
    para que queden bonitas. Igualmente hay algunos pasos de los que no
    estoy del todo convencido, habría que revisar estos. \tiny --- Luisma ---}\\
  ---------------------------------------------------------------------------
  
  Sea la curva $\gamma(t),\,\, t\in[a,b]\subset\mathbb{R}$, con curvatura
  $$K_{\gamma}(t)=\frac{det(\gamma'(t), \gamma''(t))}{||\gamma(t)'||^{3}};\,\,\,\,
  \frac{dK_{\gamma}(s)}{ds} = \frac{dK_{\gamma}(t)}{dt}
  \frac{1}{||\gamma'(t)||} = K_{\gamma}'(t) * \frac{1}{||\gamma'||} $$
  Consideramos la misma curva recorrida en sentido contrario definida por
  $$\bar{\gamma}(t) = \gamma(b+a-t),\,\, t\in[a,b]\subset\mathbb{R}$$
  Entonces $\bar{\gamma}'(t)=-\gamma'(a+b-t)$ y
  $\bar{\gamma}''(t) = \gamma''(a+b-t)$.\\ Por la definición de curvatura,
  $K_{\bar{\gamma}}(t) = \frac{det(\bar{\gamma}'(t),
    \bar{\gamma}''(t))}{||\gamma'(t)||^{3}} = \frac{det(-\gamma'(a+b-t),
    \gamma''(a+b-t))}{||\gamma'(a+b-t)||^{3}} = \frac{- det(\gamma'(a+b-t),
    \gamma''(a+b-t))}{||\gamma'(a+b-t)||^{3}} = -K_{\gamma}(a+b-t)$.\\
  Además,
  $\frac{dK_{\bar{\gamma}}(s)}{ds} = \frac{dK_{\bar{\gamma}}(t)}{dt}
  \frac{1}{||\bar{\gamma}'(t)||} = \frac{d(-K_{\gamma}(a+b-t))}{dt}
  \frac{1}{||\gamma'(a+b-t)||} = -\frac{dK_{\gamma}(a+b-t)}{dt}
  \frac{1}{||\gamma(a+b-t)'||} = K_{\gamma}'(b+a-t) *
  \frac{1}{||\gamma(a+b-t)'||}$.\\

  Luego las signaturas verifican que $sig_{\bar{\gamma}}(I) = (K_{\bar{\gamma}}(I),
  \frac{dK_{\bar{\gamma}}(I)}{ds}) = (- K_{\gamma}(I),
  \frac{dK_{\gamma}(I)}{ds})$ \\
\end{enumerate}
\end{document}
